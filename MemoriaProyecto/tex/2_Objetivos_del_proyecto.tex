\capitulo{2}{Objetivos del proyecto}

%Este apartado explica de forma precisa y concisa cuales son los objetivos que se persiguen con la realización del proyecto. Se puede distinguir entre los objetivos marcados por los requisitos del software a construir y los objetivos de carácter técnico que plantea a la hora de llevar a la práctica el proyecto.
\section{Requisitos}
En este apartado se detallarán los requisitos funcionales, marcados por la aplicación que se está desarrollando y los no funcionales, planteados durante el desarrollo de la aplicación.
\subsection{Requisitos funcionales}
En primer lugar se detallan requisitos generales que surgen a partir del planteamiento del problema y los objetivos que se desean conseguir con este proyecto. En segundo lugar se definirán los requisitos más específicos de cada subsistema.
\begin{itemize}
	\item Se desea obtener métricas de evolución de uno o varios repositorios especificando sus URLs, con el objetivo de compararlos.
	\item Las métricas que se desean calcular de un repositorio son las siguientes:
	\begin{itemize}
		\item Número total de incidencia
		\item Cambios por incidencia
		\item Porcentaje de asuntos cerrados
		\item Media de días en cerrar un asunto
		\item Media de días entre cambios
		\item Días entre primer y último cambio
		\item Rango de actividad de cambios por mes
		\item Porcentaje de pico de cambios
	\end{itemize}
\end{itemize}
\subsection{Requisitos no funcionales}