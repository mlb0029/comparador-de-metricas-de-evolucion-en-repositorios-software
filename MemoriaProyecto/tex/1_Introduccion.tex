\capitulo{1}{Introducción}

%Descripción del contenido del trabajo y del estrucutra de la memoria y del resto de materiales entregados.
El trabajo se centra en crear una aplicación web que permita, a partir de URLs de repositorios GitLab, comparar dos o varios repositorios mediante métricas de evolución.
Las métricas que se van a trabajar, en principio, son:
\begin{itemize}
	\item Número total de issues (I1)
	\item Commits por issue (I2)
	\item Porcentaje de issues cerradas (I3)
	\item Media de días en cerrar una issue (TI1)
	\item Media de días entre commits (TC1)
	\item Días que han pasado entre el primer y último commit (TC2)
	\item Ratio de actividad de commits por mes (TC3)
	\item Número de commits en el mes pico (C1)
\end{itemize}
Se pretende que el proyecto pueda ser ampliado a más gestores de repositorios como GitHub o Bitbucket y que pueda calcular más métricas de las que originalmente se han trabajado, por tanto deberá presentar un diseño que facilite el mantenimiento de la misma y añadir nuevas funcionalidades.