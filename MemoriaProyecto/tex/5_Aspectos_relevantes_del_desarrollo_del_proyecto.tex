\capitulo{5}{Aspectos relevantes del desarrollo del proyecto}

%Este apartado pretende recoger los aspectos más interesantes del desarrollo del proyecto, comentados por los autores del mismo.
%Debe incluir desde la exposición del ciclo de vida utilizado, hasta los detalles de mayor relevancia de las fases de análisis, diseño e implementación.
%Se busca que no sea una mera operación de copiar y pegar diagramas y extractos del código fuente, sino que realmente se justifiquen los caminos de solución que se han tomado, especialmente aquellos que no sean triviales.
%Puede ser el lugar más adecuado para documentar los aspectos más interesantes del diseño y de la implementación, con un mayor hincapié en aspectos tales como el tipo de arquitectura elegido, los índices de las tablas de la base de datos, normalización y desnormalización, distribución en ficheros3, reglas de negocio dentro de las bases de datos (EDVHV GH GDWRV DFWLYDV), aspectos de desarrollo relacionados con el WWW...
%Este apartado, debe convertirse en el resumen de la experiencia práctica del proyecto, y por sí mismo justifica que la memoria se convierta en un documento útil, fuente de referencia para los autores, los tutores y futuros alumnos.

\section{Definición de métricas}
%Quedaría mejor con tablas
\subsection{I1 - Número total de issues}
\begin{itemize}
	\item \textbf{Categoría}: Proceso de Orientación
	\item \textbf{Nombre}: NumeroTotalIssues
	\item \textbf{Descripción}: Número total de issues creadas en el repositorio
	\item \textbf{Propósito}: ¿Cuántas issues se han definido en el repositorio?
	\item \textbf{Fórmula}: NTI. NTI = número total de issues
	\item \textbf{Fuente de medición}: Repositorio de un gestor de repositorios
	\item \textbf{Interpretación}: NTI >= 0. Mejor valores bajos
	\item \textbf{Tipo de escala}: Absoluta
	\item \textbf{Tipo de medida}: NTI = Contador
\end{itemize}
\subsection{I2 - Commits por issue}
\begin{itemize}
	\item \textbf{Categoría}: Proceso de Orientación
	\item \textbf{Nombre}: CommitsPorIssue
	\item \textbf{Descripción}: Número de commits por issue
	\item \textbf{Propósito}: ¿Cuántos commits realizados por cada issue?
	\item \textbf{Fórmula}: CI = NTC/NTI. NTI = Numero total de issues, NTC = Número total de commits
	\item \textbf{Fuente de medición}: Repositorio de un gestor de repositorios
	\item \textbf{Interpretación}: CI >= 0, Si se acerca a 1 se definen bien las issues, si alto: no se definen bien las issues, si bajo: desarrollo del proyecto lento
	\item \textbf{Tipo de escala}: Ratio 
	\item \textbf{Tipo de medida}: NTI, NTC = Contador
\end{itemize}
\subsection{I3 - Porcentaje de issues cerradas}
\begin{itemize}
	\item \textbf{Categoría}: Proceso de Orientación
	\item \textbf{Nombre}: PorcentajeIssuesCerradas
	\item \textbf{Descripción}: Porcentaje de issues cerradas
	\item \textbf{Propósito}: ¿Qué porcentaje de issues definidas en el repositorio se han cerrado?
	\item \textbf{Fórmula}: PIC = NTIC*100/NTI. NTIC = Número total de issues cerradas, NTI = Numero total de issues
	\item \textbf{Fuente de medición}: Repositorio de un gestor de repositorios
	\item \textbf{Interpretación}: 0 <= PIC <= 100. Cuanto más alto mejor
	\item \textbf{Tipo de escala}: Ratio
	\item \textbf{Tipo de medida}: NTI, NTIC = Contador
\end{itemize}
\subsection{TI1 - Media de días en cerrar una issue}
\begin{itemize}
	\item \textbf{Categoría}: Proceso de Orientación
	\item \textbf{Nombre}: PorcentajeIssuesCerradas
	\item \textbf{Descripción}: Porcentaje de issues cerradas
	\item \textbf{Propósito}: ¿Qué porcentaje de issues definidas en el repositorio se han cerrado? 
	\item \textbf{Fórmula}: PIC = NTIC*100/NTI. NTIC = Número total de issues cerradas, NTI = Numero total de issues
	\item \textbf{Fuente de medición}: Repositorio de un gestor de repositorios
	\item \textbf{Interpretación}: 0 <= PIC <= 100. Cuanto más alto mejor.
	\item \textbf{Tipo de escala}: Ratio
	\item \textbf{Tipo de medida}: NTI, NTIC = Contador
\end{itemize}
\subsection{TC1 - Media de días entre commits}
\begin{itemize}
	\item \textbf{Categoría}: Constantes de tiempo
	\item \textbf{Nombre}: MediaDiasEntreCommits
	\item \textbf{Descripción}: Media de días que pasan entre dos commits consecutivos
	\item \textbf{Propósito}: ¿Cúanto tiempo suele pasar desde un commit hasta el siguiente?
	\item \textbf{Fórmula}: MDEC = [Sumatorio de (TCi-TCj) desde i=1, j=0 hasta i=NTC] / NTC. NTC = Número total de commits, TC = Tiempo de Commit 
	\item \textbf{Fuente de medición}: Repositorio de un gestor de repositorios
	\item \textbf{Interpretación}: MDEC > 0. Cuanto más pequeño mejor.
	\item \textbf{Tipo de escala}: Ratio
	\item \textbf{Tipo de medida}: NTC = Contador; TC = Tiempo
\end{itemize}
\subsection{TC2 - Días entre primer y último commit}
\begin{itemize}
	\item \textbf{Categoría}: Constantes de tiempo
	\item \textbf{Nombre}: DiasEntrePrimerYUltimoCommit
	\item \textbf{Descripción}: Días transcurridos entre el primer y el ultimo commit 
	\item \textbf{Propósito}: ¿Cuantos días han pasado entre el primer y el último commit?
	\item \textbf{Fórmula}: DEPUC = TC2- TC1. TC2 = Tiempo de último commit, TC1 = Tiempo de primer commit.
	\item \textbf{Fuente de medición}: Repositorio de un gestor de repositorios
	\item \textbf{Interpretación}: DEPUC >= 0
	\item \textbf{Tipo de escala}: Absoluta
	\item \textbf{Tipo de medida}: TC = Tiempo
\end{itemize}
\subsection{TC3 - Ratio de actividad de commits por mes}
\begin{itemize}
	\item \textbf{Categoría}: Constantes de tiempo
	\item \textbf{Nombre}: RatioCommitPorMes
	\item \textbf{Descripción}: Muestra el número de commits relativos al número de meses
	\item \textbf{Propósito}:¿Cuál es el número medio de cambios por mes?
	\item \textbf{Fórmula}: RCM = NTC / 12
	\item \textbf{Fuente de medición}: Repositorio de un gestor de repositorios
	\item \textbf{Interpretación}: RCM > 0. Cuanto más alto mejor
	\item \textbf{Tipo de escala}: Ratio
	\item \textbf{Tipo de medida}: NTC = Contador
\end{itemize}
\subsection{C1 - Número de commits en el mes pico}
\begin{itemize}
	\item \textbf{Categoría}: Constantes de tiempo
	\item \textbf{Nombre}: ContadorCommitsPico
	\item \textbf{Descripción}: Número de commits en el mes que más commits se han realizado en relación con el número total de commits
	\item \textbf{Propósito}: ¿Cuál es la proporción de trabajo realizado en el mes con mayor número de cambios?
	\item \textbf{Fórmula}: CCP = NCMP / NTC. NCMP = Número de commits en el mes pico, NTC = Número total de commits
	\item \textbf{Fuente de medición}: Repositorio de un gestor de repositorios
	\item \textbf{Interpretación}: 0 <= CCP <= 1. Mejor valores intermedios
	\item \textbf{Tipo de escala}: Ratio
	\item \textbf{Tipo de medida}: NCMP, NTC = Contador
\end{itemize}