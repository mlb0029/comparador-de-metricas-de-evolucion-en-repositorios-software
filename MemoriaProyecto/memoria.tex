\documentclass[a4paper,11pt,oneside]{memoir}

\usepackage{verbatim} % mlb0029 Para \begin{comment}
\setlength\epigraphwidth{.8\textwidth}
\setlength\epigraphrule{0pt}

% Castellano
\usepackage[spanish,es-tabla]{babel}
\selectlanguage{spanish}
\usepackage[utf8]{inputenc}
\usepackage[T1]{fontenc}
\usepackage{lmodern} % Scalable font
\usepackage{microtype}
\usepackage{placeins}

\usepackage{graphicx}
\usepackage{subcaption}
\usepackage[numbers,sort]{natbib}

\RequirePackage{booktabs}
\RequirePackage[table]{xcolor}
\RequirePackage{xtab}
\RequirePackage{multirow}

% Links
\PassOptionsToPackage{hyphens}{url}\usepackage[colorlinks, breaklinks]{hyperref} %\PassOptionsToPackage{hyphens}{url} para cortar url largas https://tex.stackexchange.com/questions/3033/forcing-linebreaks-in-url
\hypersetup{
	colorlinks,
	allcolors = {red}
}

% Ecuaciones
\usepackage{amsmath}

% Rutas de fichero / paquete
\newcommand{\ruta}[1]{{\sffamily #1}}

% Párrafos
\nonzeroparskip


% Imagenes
\usepackage{graphicx}
\newcommand{\imagen}[2]{
	\begin{figure}[!h]
		\centering
		\includegraphics[width=0.9\textwidth]{#1}
		\caption{#2}\label{fig:#1}
	\end{figure}
	\FloatBarrier
}

\usepackage{listings}

\newcommand{\imagenflotante}[2]{
	\begin{figure}%[!h]
		\centering
		\includegraphics[width=0.9\textwidth]{#1}
		\caption{#2}\label{fig:#1}
	\end{figure}
}



% El comando \figura nos permite insertar figuras comodamente, y utilizando
% siempre el mismo formato. Los parametros son:
% 1 -> Porcentaje del ancho de página que ocupará la figura (de 0 a 1)
% 2 --> Fichero de la imagen
% 3 --> Texto a pie de imagen
% 4 --> Etiqueta (label) para referencias
% 5 --> Opciones que queramos pasarle al \includegraphics
% 6 --> Opciones de posicionamiento a pasarle a \begin{figure}
\newcommand{\figuraConPosicion}[6]{%
  \setlength{\anchoFloat}{#1\textwidth}%
  \addtolength{\anchoFloat}{-4\fboxsep}%
  \setlength{\anchoFigura}{\anchoFloat}%
  \begin{figure}[#6]
    \begin{center}%
      \Ovalbox{%
        \begin{minipage}{\anchoFloat}%
          \begin{center}%
            \includegraphics[width=\anchoFigura,#5]{#2}%
            \caption{#3}%
            \label{#4}%
          \end{center}%
        \end{minipage}
      }%
    \end{center}%
  \end{figure}%
}

%
% Comando para incluir imágenes en formato apaisado (sin marco).
\newcommand{\figuraApaisadaSinMarco}[5]{%
  \begin{figure}%
    \begin{center}%
    \includegraphics[angle=90,height=#1\textheight,#5]{#2}%
    \caption{#3}%
    \label{#4}%
    \end{center}%
  \end{figure}%
}
% Para las tablas
\newcommand{\otoprule}{\midrule [\heavyrulewidth]}
%
% Nuevo comando para tablas pequeñas (menos de una página).
\newcommand{\tablaSmall}[5]{%
 \begin{table}
  \begin{center}
   \rowcolors {2}{gray!35}{}
   \begin{tabular}{#2}
    \toprule
    #4
    \otoprule
    #5
    \bottomrule
   \end{tabular}
   \caption{#1}
   \label{tabla:#3}
  \end{center}
 \end{table}
}

%
% Nuevo comando para tablas pequeñas (menos de una página).
\newcommand{\tablaSmallSinColores}[5]{%
 \begin{table}[H]
  \begin{center}
   \begin{tabular}{#2}
    \toprule
    #4
    \otoprule
    #5
    \bottomrule
   \end{tabular}
   \caption{#1}
   \label{tabla:#3}
  \end{center}
 \end{table}
}

\newcommand{\tablaApaisadaSmall}[5]{%
\begin{landscape}
  \begin{table}
   \begin{center}
    \rowcolors {2}{gray!35}{}
    \begin{tabular}{#2}
     \toprule
     #4
     \otoprule
     #5
     \bottomrule
    \end{tabular}
    \caption{#1}
    \label{tabla:#3}
   \end{center}
  \end{table}
\end{landscape}
}

%
% Nuevo comando para tablas grandes con cabecera y filas alternas coloreadas en gris.
\newcommand{\tabla}[6]{%
  \begin{center}
    \tablefirsthead{
      \toprule
      #5
      \otoprule
    }
    \tablehead{
      \multicolumn{#3}{l}{\small\sl continúa desde la página anterior}\\
      \toprule
      #5
      \otoprule
    }
    \tabletail{
      \hline
      \multicolumn{#3}{r}{\small\sl continúa en la página siguiente}\\
    }
    \tablelasttail{
      \hline
    }
    \bottomcaption{#1}
    \rowcolors {2}{gray!35}{}
    \begin{xtabular}{#2}
      #6
      \bottomrule
    \end{xtabular}
    \label{tabla:#4}
  \end{center}
}

%
% Nuevo comando para tablas grandes con cabecera.
\newcommand{\tablaSinColores}[6]{%
  \begin{center}
    \tablefirsthead{
      \toprule
      #5
      \otoprule
    }
    \tablehead{
      \multicolumn{#3}{l}{\small\sl continúa desde la página anterior}\\
      \toprule
      #5
      \otoprule
    }
    \tabletail{
      \hline
      \multicolumn{#3}{r}{\small\sl continúa en la página siguiente}\\
    }
    \tablelasttail{
      \hline
    }
    \bottomcaption{#1}
    \begin{xtabular}{#2}
      #6
      \bottomrule
    \end{xtabular}
    \label{tabla:#4}
  \end{center}
}

%
% Nuevo comando para tablas grandes sin cabecera.
\newcommand{\tablaSinCabecera}[5]{%
  \begin{center}
    \tablefirsthead{
      \toprule
    }
    \tablehead{
      \multicolumn{#3}{l}{\small\sl continúa desde la página anterior}\\
      \hline
    }
    \tabletail{
      \hline
      \multicolumn{#3}{r}{\small\sl continúa en la página siguiente}\\
    }
    \tablelasttail{
      \hline
    }
    \bottomcaption{#1}
  \begin{xtabular}{#2}
    #5
   \bottomrule
  \end{xtabular}
  \label{tabla:#4}
  \end{center}
}



\definecolor{cgoLight}{HTML}{EEEEEE}
\definecolor{cgoExtralight}{HTML}{FFFFFF}

%
% Nuevo comando para tablas grandes sin cabecera.
\newcommand{\tablaSinCabeceraConBandas}[5]{%
  \begin{center}
    \tablefirsthead{
      \toprule
    }
    \tablehead{
      \multicolumn{#3}{l}{\small\sl continúa desde la página anterior}\\
      \hline
    }
    \tabletail{
      \hline
      \multicolumn{#3}{r}{\small\sl continúa en la página siguiente}\\
    }
    \tablelasttail{
      \hline
    }
    \bottomcaption{#1}
    \rowcolors[]{1}{cgoExtralight}{cgoLight}

  \begin{xtabular}{#2}
    #5
   \bottomrule
  \end{xtabular}
  \label{tabla:#4}
  \end{center}
}

% TODO
\def\todo{{\color{red}Comentario de revisión...}}

\graphicspath{ {./img/} }

% Capítulos
\chapterstyle{bianchi}
\newcommand{\capitulo}[2]{
	\setcounter{chapter}{#1}
	\setcounter{section}{0}
	\chapter*{#2}
	\addcontentsline{toc}{chapter}{#2}
	\markboth{#2}{#2}
}

% Apéndices
\renewcommand{\appendixname}{Apéndice}
\renewcommand*\cftappendixname{\appendixname}

\newcommand{\apendice}[1]{
	%\renewcommand{\thechapter}{A}
	\chapter{#1}
}

\renewcommand*\cftappendixname{\appendixname\ }

% Formato de portada
\makeatletter
\usepackage{xcolor}
\newcommand{\tutor}[1]{\def\@tutor{#1}}
\newcommand{\course}[1]{\def\@course{#1}}
\definecolor{cpardoBox}{HTML}{E6E6FF}
\def\maketitle{
  \null
  \thispagestyle{empty}
  % Cabecera ----------------
\noindent\includegraphics[width=\textwidth]{cabecera}\vspace{1cm}%
  \vfill
  % Título proyecto y escudo informática ----------------
  \colorbox{cpardoBox}{%
    \begin{minipage}{.8\textwidth}
      \vspace{.5cm}\Large
      \begin{center}
      \textbf{TFG del Grado en Ingeniería Informática}\vspace{.6cm}\\
      \textbf{\LARGE\@title{}}
      \end{center}
      \vspace{.2cm}
    \end{minipage}

  }%
  \hfill\begin{minipage}{.20\textwidth}
    \includegraphics[width=\textwidth]{escudoInfor}
  \end{minipage}
  \vfill
  % Datos de alumno, curso y tutores ------------------
  \begin{center}%
  {%
    \noindent\LARGE
    Presentado por \@author{}\\ 
    en Universidad de Burgos --- \@date{}\\
    Tutor: \@tutor{}\\
  }%
  \end{center}%
  \null
  \cleardoublepage
  }
\makeatother

\newcommand{\nombre}{Miguel Ángel León Bardavío} %%% cambio de comando

% Datos de portada
\title{Comparador de métricas de evolución en repositorios software}
\author{\nombre}
\tutor{Carlos López Nozal}
\date{\today}

\begin{document}

\maketitle


\newpage\null\thispagestyle{empty}\newpage


%%%%%%%%%%%%%%%%%%%%%%%%%%%%%%%%%%%%%%%%%%%%%%%%%%%%%%%%%%%%%%%%%%%%%%%%%%%%%%%%%%%%%%%%
\thispagestyle{empty}


\noindent\includegraphics[width=\textwidth]{cabecera}\vspace{1cm}

\noindent D. Carlos López Nozal, profesor del departamento de Ingeniería Civil, Área de Lenguajes y Sistemas Informáticos.

\noindent Expone:

\noindent Que el alumno D. \nombre, con DNI 71362165L, ha realizado el Trabajo final de Grado en Ingeniería Informática titulado \textit{Comparador de métricas de evolución en repositorios software}. 

\noindent Y que dicho trabajo ha sido realizado por el alumno bajo la dirección del que suscribe, en virtud de lo cual se autoriza su presentación y defensa.

\begin{center} %\large
En Burgos, {\large \today}
\end{center}

\vfill\vfill\vfill

%% Author and supervisor
%\begin{minipage}{0.45\textwidth}
%\begin{flushleft} %\large
%Vº. Bº. del Tutor:\\[2cm]
%D. Carlos López Nozal
%\end{flushleft}
%\end{minipage}
%\hfill
%%\begin{minipage}{0.45\textwidth}
%%\begin{flushleft} %\large
%%Vº. Bº. del co-tutor:\\[2cm]
%%D. nombre co-tutor
%%\end{flushleft}
%%\end{minipage}
%\hfill
%
%\vfill

% para casos con solo un tutor comentar lo anterior
% y descomentar lo siguiente
%Vº. Bº. del Tutor:\\[2cm]
%D. nombre tutor
%Adaptación mlb0029
\begin{minipage}{1\textwidth}
\begin{center} %\large
Vº. Bº. del Tutor:\\[2cm]
D. Carlos López Nozal
\end{center}
\end{minipage}

\newpage\null\thispagestyle{empty}\newpage




\frontmatter

% Abstract en castellano
\renewcommand*\abstractname{Resumen}
\begin{abstract}
%\todo  explicar brevemente  proceso software y su relación con los repositorios de software. Métricas de evolución. ¿qué hace el trabajo? y algun resultado
%\todo Como versión inicial que necesita ser ajustado   añado idea de de un artículo  titulado Emerging topics in mining software repositories.
El proceso del software es un conjunto de actividades cuya meta es el desarrollo o evolución de software. Algunos ejemplos de estas actividades son: la especificación, el diseño, la implementación, pruebas, el aseguramiento de calidad, la configuración del proyecto, etc. 

Los repositorios de código, además de almacenar el código fuente de un proyecto software, pueden incluir sistemas que faciliten las actividades del proceso de software: sistemas de control de versiones, sistemas de seguimiento de incidencias, sistemas de revisión de código, sistemas de despliegue de ejecutables, etc. En la última década han surgido forjas de repositorios que permiten alojar múltiples proyectos, estas son útiles tanto para proyectos empresariales como para proyectos open source.

%\todo Reducir párrafo en inglés
%A software process is a set of related acti\-vi\-ties that culminates in the production of a software package: specification, design, implementation, testing, evolution into new versions, and maintenance. There are also other supporting activities such as configuration and change management, quality assurance, project management, evaluation of user experience, etc. Software repositories are infrastructures to support all these activities. They can be composed with several systems that include code change management, bug tracking, code review, build system, release binaries, wikis, forums, etc. 
Las métricas de evolución ayudan a cuantificar características de los procesos software. Un ejemplo de este tipo de medidas es el \textit{número de días de cierre}, en la que se mide el numero de días que pasan desde que se abre una incidencia hasta su cierre. Estas métricas se pueden obtener gracias a los datos estadísticos que proporcionan los repositorios. 

En este TFG se diseña un software para calcular métricas de evolución sobre distintos repositorios. En el diseño se ha optado por implementar una aplicación web en Java que toma como entrada un conjunto de repositorios públicos o privados de GitLab y calcula métricas de evolución que permiten comparar los proyectos. Además, se ha procurado un diseño extensible a otras forjas de repositorios y a nuevas métricas. La aplicación ha sido probada con Trabajos Fin de Grado presentados en la Universidad de Burgos y que han sido almacenados en repositorios públicos de GitLab.

\end{abstract}

\renewcommand*\abstractname{Descriptores}
\begin{abstract}
Métricas de evolución, repositorios de código, proceso de desarrollo de software, ciclo de vida de desarrollo de software, gestión de calidad, forjas de repositorios, comparación de proyectos software, aplicación web.
\end{abstract}

\clearpage

% Abstract en inglés
\renewcommand*\abstractname{Abstract}
\begin{abstract}

%\todo Idem resumen en español

The software process is a set of activities whose goal is the development or evolution of software. Some examples of these activities are: specification, design, implementation, testing, quality assurance, project management, etc.

The source code repositories, in addition to storing the source code of a software project, may include systems that ease the activities of the software development process: version control systems, issue tracking systems, code review systems, deployment systems, etc. Forges of source code repositories have emerged in the last decade that allow hosting multiple projects, these are useful for both business projects and open-source projects.

Evolution metrics helps to quantify features of a software development process. An example of this type of measure is the \textit {days to close an issue}, in which the number of days that pass from when an incident is opened until its closure is measured. These metrics can be obtained from the statistics provided by the source code repositories.

In this project, a software is designed to calculate evolution metrics on different source code repositories. The design has chosen to implement a web application in Java language that takes as input a set of GitLab public or private repositories and calculates evolution metrics that allow the repositories to be compared. In addition, an extensible design to other repositories forges and new metrics has been sought. The application has been tested with Final Degree Projects presented at the University of Burgos and that have been stored in public repositories of GitLab.

\end{abstract}

\renewcommand*\abstractname{Keywords}
\begin{abstract}
Evolution metrics, source code repositories, software development process, software development life cycle, quality management, forge of repositories, comparison of software projects, web application.
\end{abstract}

\clearpage

% Indices
\tableofcontents

\clearpage

\listoffigures

\clearpage

%mlb0029 - No hay tablas
%\listoftables
%\clearpage

\mainmatter
\capitulo{1}{Introducción}

%Descripción del contenido del trabajo y del estrucutra de la memoria y del resto de materiales entregados.

El desarrollo de software es un proceso muy complejo que depende de múltiples factores: equipo de desarrollo, tipo de producto software, estabilidad de los requisitos funcionales, importancia de los requisitos no funcionales como escalabilidad, seguridad, licencias, lenguaje de programación, tipo de arquitectura de computación...

Existen varias metodologías que ayudan a definir las actividades y artefactos del proceso de desarrollo de software. Los artefactos son las salidas de las actividades y el conjunto de artefactos conforman el producto software. En el caso de  Unified Process (UP) \cite{jacobson_proceso_2000} se identifican las siguientes actividades o flujos de trabajo: recolección de requisitos, diseño e implementación, pruebas, despliegue. Además en UP se añaden tres flujos de trabajo de soporte: configuración de cambios, gestión de proyecto y entorno. Estos flujos de trabajo se aplican iterativamente durante varias fases del desarrollo en cada una de las cuales se incrementa el producto software con algun artefacto resultado de la actividad. La característica de iteración e incremental es recogida en otros métodos o buenas prácticas de desarrollo ágiles\cite{noauthor_scrum_2019}: 
Scrum,eXtreme Programming,Lean...
%\todo Añadir referencias bibliográficas a libros


Una suposición subyacente de la administración de la calidad es que la calidad del proceso de desarrollo afecta directamente a la calidad de los productos a entregar \citep[pág 543]{sommerville_ingenierisoftware_2002}.
La calidad del proceso es uno de los factores que determinan la calidad del producto software, tal y como expone Sommerville en \textit{Ingeniería de software}\cite{sommerville_ingenierisoftware_2002}.

Los repositorios de código son espacios virtuales donde los equipos de desarrollo generan los artefactos colaborativos procedentes de las actividades de un proceso de desarrollo. En los repositorios además de guardar los artefactos, version final y versiones previas, se almacena la interacción de los miembros del equipo justificando el cambio de version. Dependiendo del artefacto generado se utiliza distintos sistemas: foros de comunicación, sistemas de control de versiones, sistemas de documentación, sistemas de gestión de issues, gestión de pruebas, de revisiones de calidad, de  integración y despliegue contínuo \cite{guemes-pena_emerging_2018}.

En la última década han surgido forjas de proyectos software de fácil acceso tanto para proyectos empresariales como para proyectos open source (SourceForge \footnote{\url{https://sourceforge.net/}}, Github \footnote{\url{https://github.com/}}, GitLab \footnote{\url{https://about.gitlab.com/}}, Bitbucket  \footnote{\url{https://bitbucket.org/}}).  Estas forjas suelen integrar múltiples sistemas para dar soporte a los flujos de trabajo y registrar las interacciones entre los miembros del equipo. Además dan la posibilidad de extensión funcional con sistemas de terceros para gestionar otras actividades no soportadas directamente, por ejemplo Travis CI
\footnote{\url{https://travis-ci.org/}} para gestionar la integración contínua o Codacy \footnote{\url{https://www.codacy.com/}} para gestionar las revisiones automáticas de calidad. Actualmente estas forjas han tenido una gran aceptación entre la comunidad de desarrolladores y existen muchos desarrollos de software de tendencia que las utilizan. La forjas de proyectos software proporcionan interfaces de programación específicas que permiten acceder a toda la información registrada. 
En la Fig. \ref{fig:Mintro-trendforja} se aprecia como cambia la tendencia de utilización de dichas forjas en el tiempo. Actualmente la forja predominante es claramente GitHub pero se ve un incremento en el uso de Gitlab.

\begin{figure}[!h]
	\centering
	\includegraphics[scale=0.4]{Mintro-trendforja.png}
	\caption{Comparativa de tendencia de búsqueda de Google desde 2004 con los términos de distintas forjas de proyectos software.}
	\label{fig:Mintro-trendforja}
\end{figure}



 
Parece lógico considerar como hipótesis que la calidad de un artefacto software tenga alguna relación con la manera en la que el equipo  aplica las actividades del proceso de desarrollo dentro del repositorio.  
La validación empírica de estas  hipótesis ha abierto una nueva línea de aplicación con los conjuntos de datos extraídos de los repositorios.  
Estas forjas de proyectos software están en constante evolución, tanto en sus estructuras estáticas como en sus interacciones dinámicas en los proyectos. Se registran grandes conjuntos de datos difíciles de procesar. El  desafío a la comunidad científica y empresarial  es constante mostrando un incremento en el interés en las aplicaciones que mejoren sus sistemas de decisión.

Este trabajo presenta el diseño y desarrollo de una aplicación Java Web que: calcula un conjunto de métricas de evolución del proceso,
 de proyectos software alojados en GitLab, para posteriormente, compararlos en relación con otros proyectos.

\section{Estructura de la memoria}

La memoria se estructura de la siguiente manera\footnote{\url{https://github.com/ubutfgm/plantillaLatex}}\cite{ubu_plantilla_2019}:

\begin{description}
	\tightlist
	\item[Introducción.] Introducción al trabajo realizado, estructura de la memoria y listado de materiales adjuntos.
	\item[Objetivos del proyecto.] Objetivos que se persiguen alcanzar con la realización del proyecto.
	\item[Conceptos teóricos.] Conceptos clave para comprender los objetivos, el proceso y el producto del proyecto.
	\item[Técnicas y herramientas.] Técnicas y herramientas utilizadas durante el desarrollo del proyecto.
	\item[Aspectos relevantes del desarrollo.] Aspectos destacables durante el proceso de desarrollo del proyecto.
	\item[Trabajos relacionados.] Otros proyectos de la misma naturaleza y los cuales han ayudado a la realización de este.
	\item[Conclusiones y líneas de trabajo futuras.] Conclusiones tras la realización del proyecto y posibilidades de mejora o expansión.
\end{description}

Se incluyen también los siguientes anexos:

\begin{description}
	\tightlist
	\item[Plan del proyecto software.] Planificación temporal y estudio de la viabilidad del proyecto.
	\item[Especificación de requisitos del software.] Análisis de los requisitos.
	\item[Especificación de diseño.] Diseño de los datos, diseño procedimental y diseño arquitectónico.
	\item[Manual del programador.] Aspectos relevantes del código fuente.
	\item[Manual de usuario.] Manual de uso para usuarios que utilicen la aplicación.
\end{description}
\capitulo{2}{Objetivos del proyecto}

%Este apartado explica de forma precisa y concisa cuales son los objetivos que se persiguen con la realización del proyecto. Se puede distinguir entre los objetivos marcados por los requisitos del software a construir y los objetivos de carácter técnico que plantea a la hora de llevar a la práctica el proyecto.
En este capítulo se detallarán los objetivos generales que se desean alcanzar en este proyecto, así como los objetivos más técnicos.

\section{Objetivos generales}
El objetivo general de este TFG es diseñar una aplicación web en Java que permita obtener un conjunto de métricas de evolución del proceso software \cite{ratzinger_space:_2007} a partir de repositorios de GitLab, para permitir comparar los distintos procesos de desarrollo software de cada repositorio.
La aplicación se probará con datos reales para comparar los repositorios de Trabajos Fin de Grado del Grado de Ingeniería Informática presentados en GitLab.
   
A continuación se desglosa el objetivo general  en objetivos más detallados.
\begin{itemize}
	\tightlist
	\item Se obtendrán medidas de métricas de evolución de uno o varios proyectos alojados en repositorios de GitLab.
	\item Las métricas que se desean calcular de un repositorio  son algunas de las especificadas en \textit{sPACE: Software Project Assessment in the Course of Evolution} \cite{ratzinger_space:_2007} y 
	adaptadas a los repositorios software:
	\begin{itemize}
		\tightlist
		\item Número total de incidencias (\textit{issues})
		\item Cambios(\textit{commits}) por incidencia
		\item Porcentaje de incidencias cerrados
		\item Media de días en cerrar una incidencia
		\item Media de días entre cambios
		\item Días entre primer y último cambio
		\item Rango de actividad de cambios por mes
		\item Porcentaje de pico de cambios
	\end{itemize}
	\item El objetivo de obtener las métricas es poder evaluar el estado de un proyecto comparándolo con otros proyectos de la misma naturaleza. Para ello se deberán establecer unos valores umbrales por cada métrica basados en el cálculo de los cuartiles Q1 y Q3. Además, estos valores se calcularán dinámicamente y se almacenarán en perfiles de configuración de métricas.
	\item Se dará la posibilidad de almacenar de manera persistente estos perfiles de métricas para permitir comparaciones futuras. Un ejemplo de utilidad es guardar los valores umbrales de repositorios por lenguaje de programación, o en el caso de repositorios de TFG de la UBU por curso académico.
	\item También se permitirá almacenar de forma persistente las métricas obtenidas de los repositorios para su posterior consulta o tratamiento. Esto permitiría comparar nuevos proyectos con proyectos de los que ya se han calculado sus métricas.
\end{itemize}
\section{Objetivos técnicos}
Este apartado recoge los requisitos más técnicos del proyecto.
\begin{itemize}
	\tightlist
	\item Diseñar la aplicación de manera que se puedan extender con nuevas métricas con el menor coste de mantenimiento posible. Se aplicará un diseño basado en frameworks y en patrones de diseño \citep{gamma_patrones_2002}.
	\item El diseño de la aplicación debe facilitar la extensión a otras plataformas de desarrollo colaborativo como GitHub o Bitbucket. Se aplicará un diseño basado en frameworks y en patrones de diseño \citep{gamma_patrones_2002}.
	\item Aplicar el \textit{frameworks} modelo vista controlador para separar la lógica de la aplicación para el calculo de métricas y la interfaz de usuario.
	Se aplicará un diseño basado en frameworks y en patrones de diseño \citep{gamma_patrones_2002}.
	\item Crear una batería de pruebas automáticas con cobertura del por encima del 90\% en los subsistemas de lógica de la aplicación .
	\item Utilizar una plataforma de desarrollo colaborativo que incluya un sistema de control de versiones, un sistema de seguimiento de incidencias y que permita una comunicación fluida entre el tutor y el alumno.
	\item Utilizar un sistema de integración y despliegue continuo.
	\item Diseñar una correcta gestión de errores definiendo excepciones de biblioteca y registrando eventos de error e información en ficheros de \textit{log}. 
	\item Aplicar nuevas estructuras  del lenguaje Java para el desarrollo, como son expresiones lambda. 
	\item Utilizar sistemas que aseguren la calidad continua del código que permitan evaluar la deuda técnica del proyecto.
	\item Probar la aplicación con ejemplos reales y utilizando técnicas avanzadas, como entrada de datos de test en ficheros con formato tabulado tipo CSV (\textit{comma separated values}). 	
\end{itemize}

\capitulo{3}{Conceptos teóricos}

%En aquellos proyectos que necesiten para su comprensión y desarrollo de unos conceptos teóricos de una determinada materia o de un determinado dominio de conocimiento, debe existir un apartado que sintetice dichos conceptos.
%
%Algunos conceptos teóricos de \LaTeX \footnote{Créditos a los proyectos de Álvaro López Cantero: Configurador de Presupuestos y Roberto Izquierdo Amo: PLQuiz}.
%
%\section{Secciones}
%
%Las secciones se incluyen con el comando section.
%
%\subsection{Subsecciones}
%
%Además de secciones tenemos subsecciones.
%
%\subsubsection{Subsubsecciones}
%
%Y subsecciones. 
%
%
%\section{Referencias}
%
%Las referencias se incluyen en el texto usando cite \cite{wiki:latex}. Para citar webs, artículos o libros \cite{koza92}.
%
%
%\section{Imágenes}
%
%Se pueden incluir imágenes con los comandos standard de \LaTeX, pero esta plantilla dispone de comandos propios como por ejemplo el siguiente:
%
%\imagen{escudoInfor}{Autómata para una expresión vacía}
%
%
%
%\section{Listas de items}
%
%Existen tres posibilidades:
%
%\begin{itemize}
%	\item primer item.
%	\item segundo item.
%\end{itemize}
%
%\begin{enumerate}
%	\item primer item.
%	\item segundo item.
%\end{enumerate}
%
%\begin{description}
%	\item[Primer item] más información sobre el primer item.
%	\item[Segundo item] más información sobre el segundo item.
%\end{description}
%	
%\begin{itemize}
%\item 
%\end{itemize}
%
%\section{Tablas}
%
%Igualmente se pueden usar los comandos específicos de \LaTeX o bien usar alguno de los comandos de la plantilla.
%
%\tablaSmall{Herramientas y tecnologías utilizadas en cada parte del proyecto}{l c c c c}{herramientasportipodeuso}
%{ \multicolumn{1}{l}{Herramientas} & App AngularJS & API REST & BD & Memoria \\}{ 
%HTML5 & X & & &\\
%CSS3 & X & & &\\
%BOOTSTRAP & X & & &\\
%JavaScript & X & & &\\
%AngularJS & X & & &\\
%Bower & X & & &\\
%PHP & & X & &\\
%Karma + Jasmine & X & & &\\
%Slim framework & & X & &\\
%Idiorm & & X & &\\
%Composer & & X & &\\
%JSON & X & X & &\\
%PhpStorm & X & X & &\\
%MySQL & & & X &\\
%PhpMyAdmin & & & X &\\
%Git + BitBucket & X & X & X & X\\
%Mik\TeX{} & & & & X\\
%\TeX{}Maker & & & & X\\
%Astah & & & & X\\
%Balsamiq Mockups & X & & &\\
%VersionOne & X & X & X & X\\
%}
\section{Evolución de software}
\subsection{Conceptos de evolución}
Un sistema de Gestión de Configuración del Software (SCM) es capaz de gestionar la evolución y cambios del código fuente en el tiempo \cite{berczuk_software_2002, sommerville_software_2016}.
En este apartado describiremos los conceptos básicos que utilizaremos para recoger, documentar, almacenar y recuperar los diferentes cambios que se produzcan en las entidades que formen parte nuestro sistema.
Un proyecto software está compuesto por múltiples ficheros y directorios. Llamaremos ítem a cualquier fichero o directorio cuya evolución en el tiempo está controlada por un sistema de Gestión de Configuración del Software (SCM). Una vez que estén bajo control, será posible ver su historial de cambios o revisiones y recuperar un estado anterior.
A medida que los ítems evolucionan en el tiempo, se van creando nuevas revisiones de los mismos. Como es lógico, algunos ítems sufrirán más cambios que otros a lo largo del desarrollo. El SCM almacena todas las revisiones de cada ítem, de manera que es posible volver al estado de uno de estos elementos en un momento dado.
La creación de revisiones ocurre a través de las operaciones de desprotección (check-out) y protección (check-in). Cuando se va a hacer una modificación en un ítem, éste se desprotege para editarlo. Para crear una nueva revisión del ítem de forma que se pueda recuperar su estado en este punto, se protege, lo que indica al SCM que debe almacenar los nuevos contenidos del ítem.
Al conjunto de revisiones de un ítem se le denomina historia y resume la evolución de ese ítem en el tiempo.
Una rama es un contenedor de revisiones, capaz de almacenar la evolución de los ítems como se muestra en la Ilustración 1. Las ramas pueden contener revisiones de más de un ítem. De hecho, esta es la situación más habitual.
Ilustración 1: Concepto de rama

Una etiqueta o label es el modo de marcar revisiones para poder agruparlas según un cierto criterio, que normalmente fija el usuario. Cuando se aplica una etiqueta, se crea una instantánea de la situación de los ítems en el tiempo. Más tarde esa instantánea puede ser referenciada con facilidad para identificar ese momento específico. Una etiqueta es, en definitiva, un nombre más fácil de recordar que se asigna a un conjunto particular de revisiones.
Las etiquetas se aplican siempre a las revisiones de los ítems que se encuentren actualmente en el espacio de trabajo o workspace, que es la zona donde el SCM puede mantener ítems bajo el control de versiones. A efectos prácticos, el workspace no dejará de ser un directorio en el disco.
Los ítems, sus revisiones, las ramas donde se almacenan las revisiones y las etiquetas que agrupan las revisiones se almacenan en un repositorio, que será el espacio principal donde el SCM guarda todos los objetos.


\subsection{Métricas de evolución}
El conjunto de métricas utilizados en este proyecto proceden de la Master Tesis titulada \textit{sPACE: Software Project Assessment in the Course of Evolution} \cite{ratzinger_space:_2007}.

\section{Framework de medición}
Para la implementación de las métricas se ha seguido la solución basada en frameworks propuesta en Soporte de Métricas con Independencia del Lenguaje para la Inferencia de Refactorizaciones [8]. Este framework (\ref{fig:MCTMotorMetricas}) es independiente del lenguaje y su objetivo es la reutilización en la implementación del cálculo de métricas.

\imagen{MCTMotorMetricas}{Diagrama del Framework para el cálculo de métricas con perfiles.}

\capitulo{4}{Técnicas y herramientas}

%Esta parte de la memoria tiene como objetivo presentar las técnicas metodológicas y las herramientas de desarrollo que se han utilizado para llevar a cabo el proyecto. Si se han estudiado diferentes alternativas de metodologías, herramientas, bibliotecas se puede hacer un resumen de los aspectos más destacados de cada alternativa, incluyendo comparativas entre las distintas opciones y una justificación de las elecciones realizadas. 
%No se pretende que este apartado se convierta en un capítulo de un libro dedicado a cada una de las alternativas, sino comentar los aspectos más destacados de cada opción, con un repaso somero a los fundamentos esenciales y referencias bibliográficas para que el lector pueda ampliar su conocimiento sobre el tema.
Este capitulo muestra las herramientas que se han utilizado para la alcanzar los objetivos del proyecto.
\section{Herramientas utilizadas}
\subsection{Entorno de desarrollo}
\todo En todas las herramientas describir brevemente las funcionalidades de que se utilizan concretamente en el proyecto.
\todo Se pueden acompañar con alguna captura de pantalla, porción de código y a una referencia al manual del progromador para obtener más detalle.
\todo Por ejemplo Java SE v11.0.1 Expresiones Lambda en el uso de stream. 
 \todo Apache Maven indicar brevemente el  flujo de trabajo utilizado de compilación, testing, calidad, package y despliegue 
\begin{description}
	\item[Eclipse IDE for Java EE Developers]. Entorno de programación Java para aplicaciones web. Se ha utilizado la version: 2018-09 (4.9.0).\\ Enlace a página de descarga:\\ \url{https://www.eclipse.org/downloads/}
	\item[Java SE 11 (JDK)]. \textit{Java Development Kit}. Se ha utilizado la versión  v11.0.1.\\ Enlace a página de descarga:\\ \url{https://www.oracle.com/technetwork/java/javase/downloads/index.html}
	\item[Apache Maven]. Gestor de proyectos software que ayuda en la construcción del proyecto, la generación de documentación, generación de informes, gestión de dependencias, integración con un sistema de control de versiones, etc. Se ha utilizado la versión  v3.6.0.\\ Enlace a página de descarga:\\ \url{https://maven.apache.org/download.cgi}
	\item[Apache Tomcat]. Contenedor de aplicaciones web con soporte de servlets Java. Sirve para desplegar la aplicación. Se ha utilizado la versión  v9.0.13.\\ Enlace a página de descarga:\\ \url{https://tomcat.apache.org/download-90.cgi}
\end{description}
\subsection{Logging}
\begin{description}
	\item[SLF4J]. Fachada de logging.\\ Enlace a página de descarga:\\ \url{https://www.slf4j.org/download.html}
	\item[Log4j 2]. Logger. Se ha utilizado la versión  v2.11.2.\\ Enlace a página de descarga:\\ \url{https://logging.apache.org/log4j/2.x/download.html}
\end{description}
\subsection{Pruebas}
\begin{description}
	\item[JUnit5]. Conjunto de bibliotecas para el desarrollo de pruebas unitarias. Se ha utilizado la versión  v5.3.1.\\ Enlace a página de descarga:\\ \url{https://junit.org/junit5/}
\end{description}
\subsection{Frameworks y librerías específicas para el proyecto}
\todo Además de los enlaces de la aplicación añade los enlaces de tu repositorio
\begin{description}
	\item[gitlab4j-api]. Framework de conexión a GitLab API. Se ha utilizado la versión  v4.9.14.\\ Enlace:\\ \url{https://github.com/gitlab4j/gitlab4j-api}\\
	Se ha preferido frente a timols/java-gitlab-api\footnote{\url{https://github.com/timols/java-gitlab-api}} tras realizar un estudio sobre sus métricas de evolución y una comparativa sobre la documentación. Concluyendo en que \textbf{gitlab4j-api} contiene mejor documentación, mejor evolución y a día de hoy se sigue desarrollando.
	\item[Apache Commons Math]. Librería que se utiliza para matemáticas descriptivas utilizada para el cálculo de cuartiles. Se ha utilizado la versión  v3.6.1.\\ Enlace a página de descarga:\\ \url{https://commons.apache.org/proper/commons-math/download_math.cgi}
\end{description}
\subsection{Interfaz gráfica}
\begin{description}
	\item[Vaadin]. Framework para desarrollo de interfaces Web con Java. Se ha utilizado la versión  v13.0.0\\ Enlace:\\ \url{https://vaadin.com/}
\end{description}
\subsection{CI/CD y Calidad de código}
\begin{description}
	\item[GitLab]. Plataforma de desarrollo colaborativo en la que se ha almacenado el proyecto en un repositorio Git.\\ Enlace:\\ \url{https://gitlab.com/users/sign_in}
	\item[Codacy]. Herramienta de generación automática de informes de calidad de código.\\ Enlace:\\ \url{https://www.codacy.com/}
	\item[JaCoCo]. Librería para cobertura de código en Java. Se ha utilizado la versión v0.8.3.\\ Enlace:\\ \url{https://www.eclemma.org/jacoco/}
	\item[Heroku]. Herramienta para despliegue continuo.\\ Enlace:\\ \url{https://id.heroku.com/login}
\end{description}
\subsection{Documentación}
\begin{description}
	\item[LaTeX]. Sistema de composición de textos.\\ Enlace:\\ \url{https://www.latex-project.org/}
	\item[TeXstudio]. Entorno de desarrollo de documentos LaTeX.\\ Enlace:\\ \url{https://www.texstudio.org/}
	\item[Zotero]. Herramienta de gestión de fuentes bibliográficas.\\ Enlace:\\ \url{https://www.zotero.org/}
\end{description}
\subsection{Técnicas}
A lo largo del proyecto se han utilizado numerosos patrones de diseño \cite{gamma_patrones_2002} como Singleton, Factory Method, Wrapper, Builder, Listener, etc.

Para el motor de métricas se ha utilizado como base el framework propuesto en \textit{Soporte de Métricas con Independencia del Lenguaje para la Inferencia de Refactorizaciones} \cite{marticorena_soporte_2005}. Ver figura \ref{fig:MCTMotorMetricas}.
%herramientas codacy maven, gitlab, capturas, etc, cobertura, heroku, ci, pipeline, token...
\capitulo{5}{Aspectos relevantes del desarrollo del proyecto}

%Este apartado pretende recoger los aspectos más interesantes del desarrollo del proyecto, comentados por los autores del mismo.
%Debe incluir desde la exposición del ciclo de vida utilizado, hasta los detalles de mayor relevancia de las fases de análisis, diseño e implementación.
%Se busca que no sea una mera operación de copiar y pegar diagramas y extractos del código fuente, sino que realmente se justifiquen los caminos de solución que se han tomado, especialmente aquellos que no sean triviales.
%Puede ser el lugar más adecuado para documentar los aspectos más interesantes del diseño y de la implementación, con un mayor hincapié en aspectos tales como el tipo de arquitectura elegido, los índices de las tablas de la base de datos, normalización y desnormalización, distribución en ficheros3, reglas de negocio dentro de las bases de datos (EDVHV GH GDWRV DFWLYDV), aspectos de desarrollo relacionados con el WWW...
%Este apartado, debe convertirse en el resumen de la experiencia práctica del proyecto, y por sí mismo justifica que la memoria se convierta en un documento útil, fuente de referencia para los autores, los tutores y futuros alumnos.

%Despliegue continuo - direccion de app en heroku. Sistema gratuito sirve para validar, pero no para explotar
%Diseño extensible
%Framework vaadin
%No responsive

%Comparacion de tfgs en la ubu, captura dde pantalla con comparativa tfg
Este capítulo recoge los aspectos destacables durante el desarrollo del proyecto.
\section{Motivación de la elección}
La elección de este trabajo fue motivada por su relación con la asignatura de \textit{Desarrollo Avanzado de Sistemas Software}, en la que se enseña como desarrollar software de calidad.
\section{Evolución del proyecto}
Tras la elección del proyecto se acordó definir una evolución que siga las bases del modelo Scrum. Tomando un proceso de desarrollo incrementa con revisión de las iteraciones cada dos semanas.
Se han definido sprints de dos semanas. Estas reuniones constaban de dos partes:
\begin{itemize}
	\item Revisión del sprint: En las que se revisaba el incremento generado, los problemas que hubo durante su desarrollo, las soluciones que se han implementado o que se plantean para el siguiente sprint.
	\item Planificación del siguiente sprint: Se definían las nuevas tareas.
\end{itemize}
Las primeras fases del desarrollo fueron de investigación y configuración. Luego se planteó un diseño inicial, que fue la base para la implementación de nuevas funcionalidades aunque, con el tiempo, se fue modificando el diseño inicial para adaptarlo a las nuevas funcionalidades o para resolver ciertos problemas que aparecían. En las siguiente secciones se detalla más a fondo cada una de estas etapas de desarrollo.
\section{Documentación}
La fase de documentación duró un mes y fue a la par con la etapa de configuración. 

Se recopiló información sobre trabajos relacionados como \textit{Activiti-Api}\cite{rlp0019_software_2019}, \textit{Soporte de Métricas con Independencia del Lenguaje para la Inferencia de Refactorizaciones} \cite{marticorena_soporte_2005} y \textit{sPACE: Software Project Assessment in the Course of Evolution} \cite{ratzinger_space:_2007} y se estudió en qué entornos y qué herramientas se utilizarían para el desarrollo del proyecto.

Uno de los estudios más relevantes fue la elección de un API Java que permitiese la conexión a GitLab. Había tres opciones:
\begin{itemize}
	\item Crear un framework propio de conexión a GitLab a partir de GitLab API. Añadía cierta complejidad al proyecto al tener que desarrollar otro módulo más, pero permitía poder definir las funciones que se necesitaban.
	\item Usar timols/java-gitlab-api\footnote{\url{https://github.com/timols/java-gitlab-api}}\cite{olshansky_wrapper_2019}. Al principio fue la solución que se escogió, pero posteriormente se descubrió otro API bastante mejor. La documentación es bastante pobre y la evolución del proyecto software estaba parada o no evolucionaba bien, tenían demasiadas incidencias abiertas y no ofrecía gran parte de la funcionalidad que aportaba GitLab API.
	\item  Usar \footnote{\url{https://github.com/gitlab4j/gitlab4j-api}}\cite{noauthor_gitlab4j_2019}. Es un proyecto bastante decente y, a día de hoy, sigue creciendo. Tiene un alto porcentaje de incidencias cerradas, un gran número de releases, y evolución constante. Este es el API con el que se ha desarrollado este proyecto.
\end{itemize}
Otro de las decisiones más difíciles fue la versión de Java. Recientemente se lanzó la versión de Java 11 y es la que inicialmente se utilizó en el proyecto. Ha habido bastantes problemas de compatibilidad, pero se han ido solucionando a lo largo del tiempo.
\section{Configuración del proyecto}
Ha sido una de las etapas más complicadas del proyecto, con frecuencia aparecían problemas con herramientas que se elegían por problemas de compatibilidad con otras herramientas, escasa documentación, etc.

El proyecto se iba a desarrollar en Java desde el principio. La versión más moderna que había entonces era Java 11. Al principio generó muchos problemas porque otras herramientas no la soportaban, pero al final se consiguió 
%\section{Desarrollo}
%\section{Pruebas}
\section{CI/CD}
Se han utilizado los sistemas de integracion continua y despliegue continuo de GitLab para controlar el correcto funcionamiento de la aplicación después de un cambio y para mejorar la calidad de las revisiones.
\capitulo{6}{Trabajos relacionados}

%Este apartado sería parecido a un estado del arte de una tesis o tesina. En un trabajo final grado no parece obligada su presencia, aunque se puede dejar a juicio del tutor el incluir un pequeño resumen comentado de los trabajos y proyectos ya realizados en el campo del proyecto en curso.
\section{Activiti-Api}

\epigraph{Aplicación que permite realizar un estudio del estado de un proyecto en GitHub mediante distintas métricas de evolución. Permitiendo tener una visión del ritmo de trabajo del proyecto, duración, numero de participantes y actividad.}{dba0010/Activiti-Api}

\begin{figure}[!h]
	\centering
	\includegraphics[width=0.65\textwidth]{M6_ActivitiApi_Ppal}
	\caption{Ventana principal de Activiti-Api}\label{fig:M6_ActivitiApi_Ppal}
\end{figure}

En un proyecto bastante parecido implementado como aplicación de escritorio escrita en lenguaje Java. En algunos aspectos ha sido modelo para el desarrollo de este proyecto software. Esta alojado en GitHub\footnote{\url{https://github.com/dba0010/Activiti-Api}} y se puede obtener y ejecutar. Se muestra la ventana principal en la Fig. \ref{fig:M6_ActivitiApi_Ppal}.

El aspecto que más llamó la atención para este proyecto es la forma de buscar repositorios. Permite establecer una conexión a GitHub iniciando sesión mediante usuario y contraseña o entrar en ``Modo desconectado'', como se muestra en la Fig. \ref{fig:M6_ActivitiApi_Con}. Una vez establecida la conexión permite mostrar las métricas de evolución de un proyecto, para ello hay que indicar un usuario del cual se quiere obtener un proyecto y seleccionar el proyecto de un desplegable con los proyectos de ese usuario, ver Fig. \ref{fig:M6_ActivitiApi_Proyectos}.

\begin{figure}[!h]
	\centering
	\includegraphics[width=0.65\textwidth]{M6_ActivitiApi_Con}
	\caption{Conexión a GitHub mediante Activiti-Api}\label{fig:M6_ActivitiApi_Con}
\end{figure}

\begin{figure}[!h]
	\centering
	\includegraphics[width=0.65\textwidth]{M6_ActivitiApi_Proyectos}
	\caption{Selección de un proyecto para obtener las métericas en Activiti-Api}\label{fig:M6_ActivitiApi_Proyectos}
\end{figure}

\subsection{Comparando Activiti-Api con este proyecto}

A continuación se muestran las diferencias del producto software de ``Activiti-Api'' con el producto generado en ``Comparador de métricas de evolución en repositorios software''.

\subsubsection{Interfaz}

Es una aplicación de escritorio, mientras que en este proyecto se ha optado por implementar una aplicación web.

\subsubsection{Conexión}

Para obtener información de los proyectos, Activiti-Api permite establecer una conexión a GitHub mientras que este proyecto permite establecer conexión a GitLab. Aunque este proyecto permite extender la funcionalidad a otras forjas de repositorios como GitHub.

Activiti-Api permite dos tipos de conexión: iniciar sesión a GitHub mediante usuario y contraseña o ``Modo desconectado'' (establece una conexión pública). 

Este proyecto permite iniciar sesión a GitLab mediante usuario y contraseña o por uso de un token de acceso personal. Además también permite establecer una conexión pública o directamente trabajar sin conexión sobre la aplicación. Dependiendo de la conexión escogida se limitará la funcionalidad de la aplicación. Por ejemplo, sin conexión no es posible añadir repositorios. Además se muestra al usuario de la aplicación el tipo de conexión escogida en todo momento y la información de sesión iniciada en caso de que se haya iniciado sesión en GitLab.

\subsubsection{Gestión de proyectos y evaluación de métricas}

Activiti-Api solo permite trabajar con un proyecto al mismo tiempo o comparar dos proyectos. La comparación se ha definido de forma estática durante el desarrollo del proyecto, permaneciendo invariable en tiempo de ejecución.

Este proyecto permite añadir múltiples proyectos, evaluarlos y compararlos mediante el cálculo estadístico de cuartiles para hallar los valores umbrales de cada métrica. Por tanto la comparación puede ser dinámica a partir de los proyectos que se escojan para la comparativa, aunque también se han definido unos valores umbrales predefinidos a partir de unas estadísticas obtenidas de un conjunto de datos obtenidos a partir de TFGs y publicado en GitHub \footnote{\url{https://github.com/clopezno/clopezno.github.io/blob/master/agile_practices_experiment/DataSet_EvolutionSoftwareMetrics_FYP.csv}}. Este proyecto permite gestionar perfiles de métricas con diferentes umbrales para distintos contextos de aplicación.

Activiti-Api genera un informe con los resultados de las métricas de un proyecto, varios gráficos y permite generar un informe de comparativa entre dos proyectos. Mientras que este proyecto muestra los resultados de varios proyectos en forma de tabla.

\subsubsection{Mantenibilidad y extensibilidad}

Este proyecto ha preparado un  framework para poder extender la fuente de datos, es decir las forjas. Ha sido implementado para obtener datos desde GitLab, pero es fácilmente extensible a otras forjas como GitHub. Activiti-Api no permite esta extensibilidad e incluye demasiadas dependencias con GitHub API.

Ambos proyectos siguen la solución basada en frameworks propuesta en \textit{Soporte de Métricas con Independencia del Lenguaje para la Inferencia de Refactorizaciones} \cite{marticorena_soporte_2005}. El objetivo del \textit{framework} es la reutilización en la implementación del cálculo de métricas. De hecho, Activiti-Api ha servido de ejemplo para la implementación del motor de métricas de este trabajo.

\section{Soporte de Métricas con Independencia del Lenguaje para la Inferencia de Refactorizaciones}

De este trabajo se ha basado la construcción del subsistema ``motor de métricas'', como se puede ver en la sección \ref{sect:3_3_3_FrameworkMedicion}.

\section{Software Project Assessment in the Course of Evolution -  Jacek Ratzinger}

Es de este trabajo de donde se han obtenido las métricas de control con las que se trabaja en este proyecto. Hay una explicación detallada en el apartado \ref{sect:3_3_2_MetricasControl}.
\capitulo{7}{Conclusiones y Líneas de trabajo futuras}

%Todo proyecto debe incluir las conclusiones que se derivan de su desarrollo. Éstas pueden ser de diferente índole, dependiendo de la tipología del proyecto, pero normalmente van a estar presentes un conjunto de conclusiones relacionadas con los resultados del proyecto y un conjunto de conclusiones técnicas. 
%Además, resulta muy útil realizar un informe crítico indicando cómo se puede mejorar el proyecto, o cómo se puede continuar trabajando en la línea del proyecto realizado. 
En este capítulo se exponen las conclusiones a las que se llegan después de realizar el trabajo, así como las posibles líneas de trabajo futuras.

\section{Conclusiones}

Las conclusiones que se extraen del trabajo son:

\begin{itemize}
	\item Se ha completado el objetivo general del proyecto: diseñar una aplicación web en Java que permita obtener un conjunto de métricas de evolución del proceso software \citep{ratzinger_space:_2007} a partir de repositorios de GitLab, para permitir comparar los distintos procesos de desarrollo software de cada repositorio. 
	
	Además se ha probado con datos reales a partir de otros repositorios de GitLab que se han presentado como TFG en el Grado de Ingeniería Informática en la Universidad de Burgos. Esto fue posible gracias a que la empresa \textit{Hewlett Packard SCDS} en su colaboración con TFGs con la \textit{Universidad de Burgos} organiza sus propuestas de TFGs en GitLab en grupos para organizarlos por cursos académicos. También hay que destacar la funcionalidad de añadir repositorios por grupo, lo que ha facilitado estas pruebas.
	
	\item Las métricas de evolución son tan importantes como las métricas de producto. Un software de calidad requiere de un proceso de calidad.
	\item Los repositorios y las forjas de repositorios facilitan el proceso de desarrollo del software y también son útiles para monitorizar este proceso, evaluarlo y mejorarlo, si es necesario.
	\item La automatización de las actividades de proceso, la integración continua y el despliegue continuo facilitan en gran medida la comunicación entre los miembros del equipo y el seguimiento de la evolución de la aplicación. También permiten detectar fallos tras realizarse un cambio. En este aspecto, GitLab facilita mucho estos procesos, más que otras forjas de repositorios conocidas.
	\item Se han utilizado gran parte de los conocimientos adquiridos durante el grado de Ingeniería Informática, e incluso se han afianzado y ampliado estos conocimientos.
	\item La revisión automática de calidad de código permite detectar rápidamente los defectos de diseño y corregirlos para mejorar la mantenibilidad de la aplicación y reducir la deuda técnica.
	\item La extensibilidad es un factor muy importante a tener en cuenta en el desarrollo de software, ya que siempre hay modificaciones sobre los requisitos funcionales de este y no hay un artefacto final, sino una evolución del artefacto anterior.
	\item Se ha conocido la utilidad de los badges para representar información rápida sobre el estado del proyecto.
	\item Se ha aprendido mucho sobre el uso de funciones avanzadas de Java y JUnit como las interfaces funcionales, los \textit{streams} y los test parametrizados.
	\item Se ha aprendido a valorar la funcionalidad de Maven como gestor de proyectos software, ya que ha reducido en gran medida las labores de configuración del proyecto. Aunque es cierto que esto ha tenido un alto coste de aprendizaje.
\end{itemize}

\section{Lineas de trabajo futuras}

Se definen en lista ideas que podrían realizarse en el futuro:
\begin{itemize}
	\item Extender la funcionalidad a nuevas métricas de evolución
	\item Extender las plataformas de desarrollo colaborativo a otras como GitHub, Bitbucket. Para GitHub, esta realizado en la rama github \footnote{\url{https://gitlab.com/mlb0029/comparador-de-metricas-de-evolucion-en-repositorios-software/tree/github}}. Habría que combinar las ramas y adaptar la interfaz gráfica.
	\item GitLab ofrece la posibilidad a los usuarios de tener su propia instancia de GitLab en un servidor propio. Por ahora solo se puede conectar al host ``https://gitlab.com/'', se podría ampliar esta funcionalidad permitiendo realizar una conexión a servidores propios
	\item Realizar un histórico de mediciones y almacenarlo en una base de datos
	\item Hacer que la aplicación web sea adaptativa (\textit{responsive})
	\item Internacionalizar la aplicación
	\item Los proyectos y perfiles de métricas importados y exportados se almacenan en un buffer en memoria. Mientras el proyecto sea pequeño no hay problema, pero conforme vaya creciendo habría que implementar otros sistemas de persistencia como bases de datos o ficheros.
	\item Se podría permitir seleccionar varios proyectos de la tabla de la página principal para poder gestionar varios proyectos a la vez. Por ejemplo: crear un perfil de métricas sólo con los proyectos seleccionados, evaluar solo los proyectos seleccionados, eliminar varios proyectos a la vez, volver a obtener métricas de varios proyectos al mismo tiempo.
	\item La aplicación web solo soporta una sesión, se podría preparar para poder explotarlo en un entorno multisesión.
\end{itemize}


\bibliography{bibliografia}
\bibliographystyle{plainnat}

\end{document}
